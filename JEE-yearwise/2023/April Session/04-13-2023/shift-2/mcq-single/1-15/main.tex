
\iffalse
	\title{Assignment}
	\author{EE24BTECH11015}
	\section{mcq-single}
\fi
	\item The area of the region $\cbrak{\brak{x,y} : x^2 \leq y \leq \abs{ x^2 - 4} , y \geq 1 }$ is \hfill\brak{April 2023}

\begin{multicols}{4}
\begin{enumerate}
\item $\frac{3}{4}\brak{4\sqrt{2}+1}$
\item $\frac{4}{3}\brak{4\sqrt{2}-1}$
\item $\frac{3}{4}\brak{4\sqrt{2}-1}$
\item $\frac{4}{3}\brak{4\sqrt{2}+1}$
\end{enumerate}
\end{multicols}

	\item If $\lim_{x \to 0} \frac{e^{ax} - \cos{\brak{bx}} - \frac{cxe^{-cx}}{2}}{1 - \cos{2x}} = 17
$, then $5a^2+b^2$ is equal to\hfill\brak{April 2023}

\begin{multicols}{4}
\begin{enumerate}
\item $76$
\item $72$
\item $64$
\item $68$
\end{enumerate}
\end{multicols}

\item The line, that is coplanar to the line $\frac{x + 3}{-3} = \frac{y - 1}{1} = \frac{z - 5}{5}$,is\hfill\brak{April 2023}

\begin{multicols}{4}
\begin{enumerate}
\item $\frac{x + 1}{-1} = \frac{y - 2}{2} = \frac{z - 5}{5}$
\item $\frac{x + 1}{1} = \frac{y - 2}{2} = \frac{z - 5}{5}$
\item $\frac{x - 1}{-1} = \frac{y - 2}{2} = \frac{z - 5}{4}$
\item $\frac{x + 1}{-1} = \frac{y - 2}{2} = \frac{z - 5}{4}$
\end{enumerate}
\end{multicols}

\item The plane, passing through the points $\myvec{0\\ -1\\ 2}$ and $\myvec{-1\\ 2\\ 1}$ and parallel to the line passing through $\myvec{5\\1\\-7}$ and $\myvec{1\\-1\\-1}$, also passes through the point:\hfill\brak{April 2023}

\begin{multicols}{4}
\begin{enumerate}
\item $\myvec{0\\5\\-2}$
\item $\myvec{-2\\5\\0}$
\item $\myvec{2\\0\\1}$
\item $\myvec{1\\-2\\1}$
\end{enumerate}
\end{multicols}

\item Let for a triangle $\Delta ABC$,
\begin{align*}
AB &= -2\hat{i} +\hat{j} +3\hat{k} \\
CB &= \alpha \hat{i} +\beta \hat{j} +\gamma \hat{k}\\
CA &= 4\hat{i} +3\hat{j} +\delta \hat{k}
\end{align*}
If $\delta >0$ and the area of the triangle $\Delta ABC$ is $5\sqrt{6}$ , then $CB \cdot CA $ is equal to
	
		\hfill\brak{April 2023}

\begin{multicols}{4}
\begin{enumerate}
\item $108$
\item $60$
\item $54$
\item $120$
\end{enumerate}
\end{multicols}

\item Let for A= $\mydet{ 1&&2&&3\\\alpha&&3 &&1\\1&&1&&1} , \abs{A}=2$. If
$\abs{2\text{ adj}\brak{2\text{ adj}\brak{2A}}}=32^n$, then $3n+\alpha$ is equal to\hfill\brak{April 2023}
\begin{multicols}{4}
\begin{enumerate}
\item $10$
\item $9$
\item $12$
\item $11$
\end{enumerate}
\end{multicols}

\item The range of $f\brak{x}=4{\sin}^{-1}\brak{\frac{x^2}{x^2+1}}$\hfill\brak{April 2023}

\begin{multicols}{4}
\begin{enumerate}
\item $\lsbrak {0},\rbrak{\pi}$
\item $\sbrak{0,\pi}$
\item $\lsbrak{0},\rbrak{2\pi}$
\item $\sbrak{0,2\pi}$
\end{enumerate}
\end{multicols}

\item Let $a_1, a_2, a_3, \dots$ be a G. P. of increasing positive numbers. Let the sum of its $6^{th}$ and $8^{th}$ terms be 2 and the product of its $3^{rd}$ and $5^{th}$ terms be $\frac{1}{9}$. Then $6\brak{a_2 + a_4}\brak{ a_4 + a_6}$ is equal to\hfill\brak{April 2023}

\begin{multicols}{4}
\begin{enumerate}
\item $2$
\item $3$
\item $3\sqrt{3}$
\item $2\sqrt{2}$
\end{enumerate}
\end{multicols}

\item If the system of equations
\begin{align*}
2x+y -z&=5\\
2x-5y+\lambda z&=\mu\\
x+2y-5z&=7
\end{align*}
has infinitely many solutions, then ${\brak{\lambda+\mu}}^2+{\brak{\lambda-\mu}}^2$ is equal to

\hfill\brak{April 2023}

\begin{multicols}{4}
\begin{enumerate}
\item $904$
\item $916$
\item $912$
\item $920$
\end{enumerate}
\end{multicols}

\item The statement $\brak{p\land\brak{\sim q}}\lor\brak{\brak{\sim p}\land q}\lor\brak{\brak{\sim p}\land \brak{\sim q}}$ is equivalent to .

	\hfill\brak{April 2023}

\begin{multicols}{4}
\begin{enumerate}
\item $\brak{\sim p}\lor\brak{\sim q}$
\item $\brak{\sim p}\land\brak{\sim q}$
\item $p \lor\brak{\sim q}$
\item $p \lor q$
\end{enumerate}
\end{multicols}

\item Let $S=\cbrak{z \in C:\Bar{z}=\iota\brak{z^2+\text{Re}\brak{\Bar{z}}}}$ . Then $\sum_{z\in S}^{} {\abs{z}}^2$

is equal to \hfill\brak{April 2023}

\begin{multicols}{4}
\begin{enumerate}
\item $4$
\item $\frac{7}{2}$
\item $3$
\item $\frac{5}{2}$
\end{enumerate}
\end{multicols}

\item Let $\alpha , \beta$ be the roots of the equation $x^2- \sqrt{2}x+2=0$, Then $\alpha^{14}+\beta^{14}$ is equal to
	
	\hfill\brak{April 2023}

\begin{multicols}{4}
\begin{enumerate}
\item $-128\sqrt{2}$
\item $-64\sqrt{2}$
\item $-128$
\item $-64$
\end{enumerate}
\end{multicols}

\item Let $\abs{\Vec{a}}=2,\abs{\Vec{b}}=3$ and the angle between the vectors $\Vec{a} \text{ and } \Vec{b} \text{ be } \frac{\pi}{4}$. Then ${\abs{\brak{\Vec{a}+2\Vec{b}}\times\brak{2\Vec{a}-3\Vec{b}}}}^2$ is equal to\hfill\brak{April 2023}

\begin{multicols}{4}
\begin{enumerate}
\item $482$
\item $841$
\item $882$
\item $441$
\end{enumerate}
\end{multicols}

\item The value of $\frac{e^{-\frac{\pi}{4}} + \int_0^{\frac{\pi}{4}} e^{-x} \tan^{50}x \, dx}{\int_0^{\frac{\pi}{4}} e^{-x} \brak{ \tan^{49}x + \tan^{51}x } \, dx}$ is\hfill\brak{April 2023}

\begin{multicols}{4}
\begin{enumerate}
\item $25$
\item $51$
\item $50$
\item $49$
\end{enumerate}
\end{multicols}

\item The coefficient of $x^5$ in the expansion of ${\brak{2x^3-\frac{1}{3x^2}}}^5$ is\hfill\brak{April 2023}

\begin{multicols}{4}
\begin{enumerate}
\item $\frac{80}{9}$
\item $8$
\item $9$
\item $\frac{26}{3}$
\end{enumerate}
\end{multicols}



